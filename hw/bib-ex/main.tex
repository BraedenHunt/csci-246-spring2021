\documentclass{article}
\usepackage{../fasy-hw}

\author{Prof.~Fasy}
\collab{n/a}

\begin{document}

\paragraph{Citation Option 1}

The first way to provide your citations is just to list them at the end of the
homework assignment.  For example:

My sources are:
\begin{enumerate}
    \item Brooks, Fred P. The Mythical Man-Month. Addison-Wesley, 1975.
    \item Wikiepedia article: \url{https://en.wikipedia.org/wiki/Fred_Brooks}
    \item \todo{add a new `item' for each new source}
\end{enumerate}

\newpage
\paragraph{Citation Option 2}

One way to add citations is to put the citation information in a
footnote.\footnote{
Brooks, Fred P. The Mythical Man-Month. Addison-Wesley, 1975.
    }
Whenever you have something that needs a citation, you can add another footnote
like this.\footnote{\todo{Another citation goes here.}}

\newpage
\paragraph{Citation Option 3}

By default, we have it setup so that LaTex will look for bibentries in
biblio.bib.   For example, I have a signed copy of
\emph{The Mythical Man-Month}~\cite{brooks1975mythical} in
German! (Now, look at the next page of the compiled PDF).
To find a starting point for bib entries, I recommend searching for the resource
on Google scholar (\url{https://scholar.google.com}).  When you find the
article, click the quote symbol under it, then the hyperlink for BibTex (and,
ask if you encounter any issues). Of course,
the \href{https://en.wikibooks.org/wiki/LaTeX/Bibliography_Management}{LaTex
Wikibook} is often
helpful.

%% ... the bibliography
\newpage
\bibliographystyle{acm}
\bibliography{biblio}

\end{document}
