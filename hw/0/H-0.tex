\documentclass{article}
\usepackage{../fasy-hw}
\usepackage{ wasysym }

%% UPDATE these variables:
\renewcommand{\hwnum}{0}
\title{Discrete Structures, Homework 0}
\author{Braeden Hunt}
\collab{n/a}
\date{due: 15 January 2021}

\begin{document}

\maketitle

This homework assignment should be
submitted as a single PDF file both to D2L and to Gradescope.

General homework expectations:
\begin{itemize}
    \item Homework should be typeset using LaTex.
    \item Answers should be in complete sentences and proofread.
    \item You will not plagiarize.
    \item List collaborators at the start of each question using the
        \texttt{collab} command.
\end{itemize}

% ============================================
% ============================================
\nextprob{Getting to Know You}
\collab{n/a}
% ============================================
% ============================================

Answer the following questions:
\begin{enumerate}
    \item What is your elevator pitch?  Describe yourself in 1-2
        sentences. 
	\paragraph{Answer} I am a first year Computer Science student at Montana State University. I come from Missoula, Montana and want to have a career in robotics.

    \item What was your favorite college class so far, and why?
        \paragraph{Answer} I really enjoyed Social and Ethical Issues in Computer Science, as it involved lots of ethical and moral issues that didn't have clear answers. We had lots of opportunities for debate.

    \item What was your least favorite college class so far, and why?
        \paragraph{Answer} My least favorite class so far has been Web Design. I \textit{heavily} dislike using CSS as it doesn't make much sense to me. 

    \item Why are you interested in taking this course? (If your answer is
        `because I am required to by my major/minor', perhaps answer the
        alternative question: Why are you in your major?)
	\paragraph{Answer} I'm in the course because it is requirement of the CS pathway. I have always enjoyed working with computers, and I hope to have a career in robotics, so computer science was the way to go for me. 

    \item What is your biggest academic or research goal for this semester (can
        be related to this course or not)?
        \paragraph{Answer} My biggest concern is maintaining my 4.0 GPA for another semester.

    \item What do you want to do after you graduate?
        \paragraph{Answer} I've mentioned it a couple of times, but I want to work in the robotics field.

    \item What was the most challenging aspect of blended or online courses?
	\paragraph{Answer} The hardest part of online courses is trying to complete labs online instead of in person.

    \item What do you like about blended or online courses?
	\paragraph{Answer} I enjoy asynchronous learning as I can pick my own schedule that best fits my needs.

\end{enumerate}

% ============================================
% ============================================
\nextprob{Administrative Tasks}
\collab{n/a}
% ============================================
% ============================================

Please do the following:
\begin{enumerate}
    \item Write this homework in LaTex. This will not be strictly enforced for
        this homework, but it is strongly encouraged.  Future homeworks will not
        be graded if they are not typeset in LaTex.
    \item Update your photo on D2L to be a recognizable headshot of you.
    \item Sign up for the class discussion board.
\end{enumerate}
	\paragraph{Answer} I have does these things.


% ============================================
% ============================================
\nextprob{Plagiarism}
\collab{n/a}
% ============================================
% ============================================


In this class, please properly cite all resources that you use.  To refresh your
memory on what plagiarism is, please complete the plagiarism tutorial found
here: \url{http://www.lib.usm.edu/plagiarism_tutorial}.  If you have observed
plagiarism or cheating in a classroom (either as an instructor or as a student),
explain the situation and how it made you feel.  If you have not experienced
plagiarism or cheating or if you would prefer not to reflect on a personal
experience, find a news article about plagiarism or cheating and explain how you
would feel if you were one of the people involved.

	\paragraph{Answer} In 2014, CNN fired a news editor for having over 100 instances of plagerism. 
I'm sure that all journalists, both the ones that she plagerized, and the ones 
that she worked with were deeply hurt by it. She was paid for submitting work
that other people did. The people should be upset that she took credit and pay 
for her work. Her coworkers should also be upset because they work hard to
create original work, and she was getting paid to copy others.

\parbox[t]{\linewidth}{\hangindent=5mm \noindent{Wemple, Erik. “CNN Fires News Editor Marie-Louise Gumuchian for Plagiarism [Updated].” \textit{The Washington Post}, WP Company, 3 May 2019, www.washingtonpost.com/blogs/erik-wemple/wp/2014/05/16/cnn-fires-news-editor-marie-louise-gumuchian-for-plagiarism/.}}

% ============================================
% ============================================
\nextprob{Exams}
\collab{n/a}
% ============================================
% ============================================

I am exploring various options for exams for this semester: take-home,
in-person, synchronous online.  If you have any comments about what worked or
did not work in previous semesters with respect to classes in blended and online
settings, please share that here.

	\paragraph{Answer} I have no comments.




% ============================================
% ============================================
\nextprob{Terminology}
\collab{n/a}
% ============================================
% ============================================

Sometimes concepts are taught more than once throughout the curriculum.  Each
time you encounter a concept, your understanding of it is deepened.
For each of the terms or statements below, describe in your own words what they
mean.  This will not be graded for correctness, just whether you have done it or
not.  Answering these to the best of your ability will help the instructor and
TA understand the base knowledge of the students in this class.
I encourage you to meet with a partner or two to refresh yourself on what these
terms mean (if you do, be sure to update the \texttt{collab} command
above!).  However, please keep the web searches to a minimum for this one!  It
is acceptable to answer `I have not heard of this term' or `I have heard of
this, but do not remember what it means.'
\begin{enumerate}
    \item $f(n)$ is $O(n^2)$.
	\paragraph{Answer} The time complexity of the functon f(n) is proportional to the square of n.
    \item $f(n)$ is $O(g(n))$.
	\paragraph{Answer} The time complexity of the function f(n) is proportional to g(n).
    \item $f(n)$ is $\Omega(n^3)$.
	\paragraph{Answer} I have not heard of this term
    \item $f(n)$ is $\Theta(n\log n)$.
	\paragraph{Answer} I have not heard of this term.
    \item Binomial Coefficients
	\paragraph{Answer} I have not heard of this term.
    \item Four Color Theorem
	\paragraph{Answer} The minimum number of colors it takes to color a map of a planar surface subdivided into areas such that no two adjacent areas have the same color.
    \item Graph
	\paragraph{Answer} In Graph Theory, a graph is a collective of nodes and edges that connect nodes.
    \item Modus Ponens
	\paragraph{Answer}  I have not heard of this term.
    \item Proof by Counter-example
	\paragraph{Answer} A mathematical proof that shows that either that a statement is false by showing an example that is can be true, or that a statement is true by showing that the opposite is impossible.
    \item Proof by Example
	\paragraph{Answer} A mathematical proof that shows that a statement can be true by showing an example of it.
    \item Proof by Induction
	\paragraph{Answer} I have not heard of this term.
    \item Recurrence Relation
	\paragraph{Answer} I have not heard of this term.
    \item Recursive Algorithm
	\paragraph{Answer} An algorithm that splits a problem into smaller and smaller parts, until it reaches a base case. The algorithm calls itself.
    \item Searching Algorithms
	\paragraph{Answer} An algorithm that searches a set for certain values.
    \item Sorting Algorithms
	\paragraph{Answer} An algorithm that orders a set of data based on a certain metric.
    \item Tree
	\paragraph{Answer} A data structure that can be in the shape of a tree. It can be used for binary searches, as well as sorting.
\end{enumerate}

% ============================================
% ============================================
\nextprob{Real Numbers}
\collab{n/a}
% ============================================
% ============================================

Review the Properties of Real Numbers in Appendix A.  If any are unfamiliar or
confusing, please post a question in the group discussion board.  In the
write-up, write the following: `I have reviewed all properties of real numbers
in Appendix A.`

	\paragraph{Answer} I have reviewed all properties of real numbers in Appendix A.


% ============================================
% ============================================
\nextprob{Georg Cantor}
\collab{n/a}
% ============================================
% ============================================

Write a short (1-2 paragraph) biography of Georg Cantor.
\textbf{In your own words}, describe who they are and why they are important in
the history of computer science.  If you use external resources, please provide
proper citations.

\paragraph{Answer} 

% ============================================

Georg Cantor was a German mathematician born in 1845. His teachers had a great 
influence on him, but one of them eventually opposed him and his work. His founding 
of set theory was controversal in mathematics at the time ("Georg Cantor"). However, this theory was 
foundational to computer science. Set Theory is useful for describing logic and 
reasoning in mathematics (Winskel), and all of mathematics can be developed with just set 
theory and logic (Epp). Set theory can therefore be used as a tool to formalize computer 
science as computer science relies heavily on logic and reason.

\parbox[t]{\linewidth}{\hangindent=5mm \noindent{“Georg Cantor.” \textit{Encyclopædia Britannica}, Encyclopædia Britannica, Inc., 2 Jan. 2021, www.britannica.com/biography/Georg-Ferdinand-Ludwig-Philipp-Cantor. }}

\parbox[t]{\linewidth}{\hangindent=5mm \noindent{Winskel, Glynn. “Set Theory for Computer Science.” \textit{University of Cambridge}, 11 Oct. 2010, www.cl.cam.ac.uk/~gw104/STfCS2010.pdf. }}

% ============================================

\end{document}

