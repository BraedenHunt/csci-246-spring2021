\documentclass{article}
\usepackage{../fasy-hw}

%% UPDATE these variables:
\renewcommand{\hwnum}{8}
\title{Discrete Structures, Homework \hwnum}
\author{Braeden Hunt (Tinnittin)}
\collab{n/a}
\date{due: 30 April 2021}

\begin{document}

\maketitle

This homework assignment should be
submitted as a single PDF file both to D2L and to Gradescope.

General homework expectations:
\begin{itemize}
    \item Homework should be typeset using LaTex.
    \item Answers should be in complete sentences and proofread.
    \item You will not plagiarize, nor will you share your written solutions
        with classmates.
    \item List collaborators at the start of each question using the \texttt{collab} command.
    \item Put your answers where the \texttt{todo} command currently is (and
        remove the \texttt{todo}, but not the word \texttt{Answer}).
\end{itemize}


% ============================================
% ============================================
\collab{n/a} \nextprob{Student Honor Code}
% ============================================
% ============================================

The final exam will be a timed exam.  You can either use LaTex to typeset or you
can scan handwritten answers (if you do the latter, please write legibly and be
sure that your scan is a clear one).  Remember that you must abide by the
student code of conduct.  Specifically, for this exam, you cannot use external
resources (other than your notes, the miro boards, and your submitted HW) or by
exceeding your limited time.  If cheating is detected in this final exam, the
exam will be an automatic F and there may be additional consequences.

Here, please explain why the code of conduct is important for the integrity of
your degree.

\paragraph{Answer}

In the physical world, a degree is just a piece of paper and maybe a record in some university database. But we as a society assign it value. We recognize the difficulty, effort, and process that it takes to earn it. It is similar to currency. A dollar is a piece of paper that we all have given value to. However, we don't give any value to counterfeit or fake dollars. If someone is giving out or paying others money and that money turns out to be fake, then no one will trust any of the money that originally came from that person. The same applies to degrees. If someone faked their degree by cheating their way through the program and still got the degree, then that degree is fake. People find out, and then they don't trust the university. This invalidates all the degrees given out by the university. Cheating in college not only harms the cheater, but the university and all the past, present, and future students of the university. By instating a code of conduct, universities reduce cheating and prevent cheaters from obtaining degrees. This prevention protects the validity of all their students' degrees.



% ============================================
% ============================================
\collab{Ben Kiehn (in class)} \nextprob{Probability}
% ============================================
% ============================================

What is the expected value of the following? Justify.

\begin{enumerate}
    \item Face value when selecting a card from a deck.  Assume that the deck of
        cards is a standard 52 card deck, the ordinal cards are worth their face
        value, the face cards (Jacks, Queens, Kings) are
        worth 10 points, and Aces are worth 15 points.
        
        \paragraph{Answer} We can ignore the suite of the card, as it doesn't affect the point value. This allows us to simply look at the 13 cards of each suite. We sum up the total point values for each card and then divide by 13 to get the expected value.
        
        $E(x) = \frac{1}{13}(A_{value} + 1_{value} + 2_{value} + 3_{value} + 4_{value} + 5_{value} + 6_{value} + 7_{value} + 8_{value} + 9_{value} + 10_{value} + J_{value} + Q_{value} + K_{value}$
        
        $E(x) = \frac{1}{13}(15 + 2 + 3 + 4 + 5 + 6 + 7 + 8 + 9 + 10 + 10 + 10 + 10) = \frac{99}{13} \approx 7.6$
        
    \item Rolling a six-sided die.  Here, the value of a roll is the square of
        the number that appears at the top.  Assume that the die is fairly
        weighted.
        
        \paragraph{Answer} We can get the expected value by summing all the possible values and dividing by 6.
        
        $E(x) = \frac{1}{6}(1_{value} + 2_{value} + 3_{value} + 4_{value} + 5_{value} + 6_{value})$
        
        $E(x) = \frac{1}{6}(1 + 4 + 9 + 16 + 25 + 6) = \frac{91}{6} \approx 15.2$
        
    \item Playing the following game: I flip a fair coin, and you select
        heads/tails while it is in the air.  If you call the correct side,
        you get 10 points.  If not, you roll a die.  If it lands as a
        multiple of three, you get -25 points.
        If it lands on the 5, then you get 100 points.  Otherwise, no points are
        awarded nor deducted.
        
        \paragraph{Answer} We first can write an equation for the expected value of the coin flip by looking at the two possible outcomes. We also need to find the expected value of the dice roll and plug that into the expected value of the coin flip to get the expected value of the entire game.
        
        $E(x) = \frac{1}{2}(Right_{value} + Wrong_{value})$
        
        $E(x) = \frac{1}{2}(10 + E_{dice}(x))$
        
        $E_{dice}(x) = \frac{1}{6}(1_{value} + 2_{value} + 3_{value} + 4_{value} + 5_{value} + 6_{value})$
        
        $E_{dice}(x) = \frac{1}{6}(0 + 0 - 25 + 0 + 100 - 25) = \frac{25}{3} \approx 8.333$
        
        $E(x) = \frac{1}{2}(10 + \frac{25}{3}) = \frac{55}{6} \approx 9.1667$
        
\end{enumerate}

% ============================================
% ============================================
\collab{n/a} \nextprob{Asymptotic Notation}
% ============================================
% ============================================

Use the definition of big-Theta to prove that $f \colon \R \to \R$ defined by
$f(x) = 3x^2+x$ is $\Theta(x^2)$.

\paragraph{Answer}

Let $f(x) = 3x^2+x$ and $g(x) = x^2$. Let $n_0 = 4$. Then, $\forall x \geq n_0$, the following holds:

$f(x) = 3x^2+x \leq 3x^2 + x + (2\sqrt{3}-1)x + 1 = 3x^2 + 2\sqrt{3}x + 3 = (\sqrt{3}x+1)^2$ because we added a positive number ($(2\sqrt{3}-1)x + 1$ is positive because $x$ is positive).

Note that $\sqrt{3}x + 1 \leq 2x$  $\forall x >= 1/(2-\sqrt{3})$

Therefore, by squaring both sides, we get $(\sqrt{3}x+1)^2 \leq 4x^2$.
Since $f(x) < (\sqrt{3}x+1)^2$, $f(x) < 4x^2 = 4g(x)$

Thus $f(x) \leq 4g(x)$ $\forall x \geq n_0$.

Furthermore, since $x > 0$ (as $4 > 0$),

$3x^2+x > 3x^2 = 3g(x)$

Thus, $f(x) > 3g(x)$, which also means $f(x) \geq 3g(x)$.

Combining this with the previous inequality, we get:

$3g(x) \leq f(x) \leq 4g(x)$ which means $f(x) \in \Theta(g(x))$ or more precisely, $f(x) \in \Theta(x^2)$, as was to be shown.




% ============================================
% ============================================
\collab{n/a} \nextprob{Class Participation}
% ============================================
% ============================================

What grade do you think you deserve for class participation? (You can give this
as a letter grade or as a number grade on a 0-100 scale). Please justify by
including approximate number or percent of class periods missed, describing how
you participate in the breakout rooms, and describing any extenuating
circumstances that we should account for.

\paragraph{Answer}

I strongly believe that I deserve an A+ for class participation. I attended every class without missing a single day. During every breakout session, I spoke and worked with my group members, most of the time being the one walking through the solutions for members who didn't understand how we got to the correct answer. I was also active outside of the breakout rooms, answering questions posed during the lecture in the chat or in Discord. I also helped my other classmates answer questions about assignments and problems in the class Discord, making sure they understood what they needed to do for a problem or nudging them in the right direction towards the solution. I'm not sure of anyway that I could have participated more in class, so that is why I believe I deserve an A+.


% ============================================
% ============================================
\collab{n/a}
\nextprob{Biography}
% ============================================
% ============================================

Write a short (1-2 paragraph) biography of a mathematician or computer scientist
mentioned in one of the two course texts (that you have not yet written a
biography for yet).
\textbf{In your own words}, describe who they are and why they are important in
the history of computer science.

If you use external resources, please provide
proper citations. If you do not use external sources, please write ``I did not
use any sources to write this biography'' as the last sentence of the
biography.

\paragraph{Answer}

Alan Turing (mentioned in Epp 12.2) was born in London in 1912. He was a British mathematician. In the 1930s, Turing worked on a famous problem, the Entscheidungsproblem. It sought an effective method for determining exactly which statements are provable within a given system. Turing proved that there was no solution to it. He used what he called the universal Turing machine, a theoretical computer that could compute everything that was humanly computable. During World War II, Turing worked on breaking the Enigma cipher, a German machine that ciphered radio communications. The machine that Turing built was so great that it was able to decipher 84,000 messages a month at its peak. That's one message every 30 seconds. After the war, he worked on digital computers and even wrote the first ever programming manual. He also did founding work in artificial intelligence, coming up with the idea that human brains are trained digital computers. He also came up with the Turing test, a test still used today to study A.I.

Unfortunately, Turing was prosecuted for being homosexual and was forced to receive hormone therapy. Later, he was found dead from cyanide poisoning. It isn't know exactly what caused this. Some speculate suicide but there isn't much evidence in his mental state for it. He could have died from inhaling cyanide fumes in the lab near his bedroom, or he could have been killed by the secret service because of his knowledge of code breaking and his ``criminal'' record.

Copeland, B. (2020, June 19). \textit{Alan Turing. Encyclopedia Britannica.} \url{https://www.britannica.com/biography/Alan-Turing}

% %% ... the bibliography
% \newpage
% \bibliographystyle{acm}
% \bibliography{biblio}

\end{document}

