\documentclass{article}
\usepackage{../fasy-hw}

%% UPDATE these variables:
\renewcommand{\hwnum}{7}
\title{Discrete Structures, Homework \hwnum}
\author{\todo{Your Name Here} (\todo{your discord handle here})}
\collab{n/a}
\date{due: 30 April 2021}

\begin{document}

\maketitle

This homework assignment should be
submitted as a single PDF file both to D2L and to Gradescope.

General homework expectations:
\begin{itemize}
    \item Homework should be typeset using LaTex.
    \item Answers should be in complete sentences and proofread.
    \item You will not plagiarize, nor will you share your written solutions
        with classmates.
    \item List collaborators at the start of each question using the \texttt{collab} command.
    \item Put your answers where the \texttt{todo} command currently is (and
        remove the \texttt{todo}, but not the word \texttt{Answer}).
\end{itemize}


% ============================================
% ============================================
\collab{\todo{}} \nextprob{Student Honor Code}
% ============================================
% ============================================

The final exam will be a timed exam.  You can either use LaTex to typeset or you
can scan handwritten answers (if you do the latter, please write legibly and be
sure that your scan is a clear one).  Remember that you must abide by the
student code of conduct.  Specifically, for this exam, you cannot use external
resources (other than your notes, the miro boards, and your submitted HW) or by
exceeding your limited time.  If cheating is detected in this final exam, the
exam will be an automatic F and there may be additional consequences.

Here, please explain why the code of conduct is important for the integrity of
your degree.

\paragraph{Answer}

\todo{your answer here}



% ============================================
% ============================================
\collab{\todo{}} \nextprob{Probability}
% ============================================
% ============================================

What is the expected value of the following? Justify.

\begin{enumerate}
    \item Face value when selecting a card from a deck.  Assume that the deck of
        cards is a standard 52 card deck, the ordinal cards are worth their face
        value, the face cards (Jacks, Queens, Kings) are
        worth 10 points, and Aces are worth 15 points.
    \item Rolling a six-sided die.  Here, the value of a roll is the square of
        the number that appears at the top.  Assume that the die is fairly
        weighted.
    \item Playing the following game: I flip a fair coin, and you select
        heads/tails while it is in the air.  If you call the correct side,
        you get 10 points.  If not, you roll a die.  If it lands as a
        multiple of three, you get -25 points.
        If it lands on the 5, then you get 100 points.  Otherwise, no points are
        awarded nor deducted.
\end{enumerate}

% ============================================
% ============================================
\collab{\todo{}} \nextprob{Asymptotic Notation}
% ============================================
% ============================================

Use the definition of big-Theta to prove that $f \colon \R \to \R$ defined by
$f(x) = 3x^2+x$ is $\Theta(x^2)$.

\paragraph{Answer}

\todo{your answer here}


% ============================================
% ============================================
\collab{\todo{}} \nextprob{Class Participation}
% ============================================
% ============================================

What grade do you think you deserve for class participation? (You can give this
as a letter grade or as a number grade on a 0-100 scale). Please justify by
including approximate number or percent of class periods missed, describing how
you participate in the breakout rooms, and describing any extenuating
circumstances that we should account for.

\paragraph{Answer}

\todo{your answer here}


% ============================================
% ============================================
\collab{\todo{}}
\nextprob{Biography}
% ============================================
% ============================================

Write a short (1-2 paragraph) biography of a mathematician or computer scientist
mentioned in one of the two course texts (that you have not yet written a
biography for yet).
\textbf{In your own words}, describe who they are and why they are important in
the history of computer science.

If you use external resources, please provide
proper citations. If you do not use external sources, please write ``I did not
use any sources to write this biography'' as the last sentence of the
biography.

\paragraph{Answer}

\todo{your answer here}

% %% ... the bibliography
% \newpage
% \bibliographystyle{acm}
% \bibliography{biblio}

\end{document}

