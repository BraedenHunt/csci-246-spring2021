\documentclass{article}
\usepackage{fasy-hw}

%% UPDATE these variables:
\renewcommand{\hwnum}{4}
\title{Discrete Structures, Homework \hwnum}
\author{{Robert Van Spyk} ({Robert Van Spyk 5586})}
\collab{n/a}
\date{due: 5 March 2021}

\begin{document}

\maketitle

This homework assignment should be
submitted as a single PDF file both to D2L and to Gradescope.

General homework expectations:
\begin{itemize}
    \item Homework should be typeset using LaTex.
    \item Answers should be in complete sentences and proofread.
    \item You will not plagiarize.
    \item List collaborators at the start of each question using the \texttt{collab} command.
    \item Put your answers where the \texttt{todo} command currently is (and
        remove the \texttt{todo}, but not the word \texttt{Answer}).
\end{itemize}


% ============================================
% ============================================
\collab{n/a} \nextprob{Good Proofs}
% ============================================
% ============================================

Look through proofs in this textbook, or other books / papers.  Define five
qualities that you think are common among good proofs. Provide citations to
examples.


\paragraph{Answer}

{Five Qualities that are common among good proofs: }


{Clarity,good proofs communicate their ideas with as much detail as necessary.}

{Logical, the logic behind the proof is sound and mathematically or logically correct.}

{Professional, correct notations, punctuation, and grammer is used.}

{Breif, good proofs use consise explanations.}

{Significant, the proof reveals important ideas related to the topic.}


{Citation to explamples: "Discrete Mathematics: An Open Introduction by Oscar Levin section 3.2" 2021)


% ============================================
% ============================================
\collab{n/a} \nextprob{Max of a Subset}
% ============================================
% ============================================

Let $(B,\leq)$ be a totally ordered finite set. Prove the following
statement: For all nonempty subsets $A \subseteq B$, the following inequality
holds: $\max(A) \leq \max(B)$.

\paragraph{Answer}

{Let x be the upper bound of N, then x is greater than or equal to max N}

{Using the contrapositive of the inequality: If x is less than max N it can't be the upper bound}

{If A is a subset of B and x is the upper bound of B and x is the upper bound of A. We get:}

{For all of a in the set A, for all of a in set B, x is greater than or equal to a because of it's the upper bound of B.}

{Therefore it is the upper bound of A}


{Which means,}


{x= max B is the upper bound for B}

{x is the upper bound for A subset B}

{max B is equal to x greater than or equal to max A}

% ============================================
% ============================================
\collab{n/a} \nextprob{Fibonacci}
% ============================================
% ============================================

The Fibonacci numbers are defined as follows:
$$
    F_i = \begin{cases}
            1 & i \in \{1,2\} \\
            F_{i-1}+F_{i-2} & \text{otherwise}
          \end{cases}
$$

Prove $\sum_{i=1}^n F_i = F_{n+2}-1$.

\paragraph{Answer}

{Given the sumation of F of i from i=1 to n, we must prove that this equates to F sub(n+2) - 1 }

{Firstly lets prove the first piece of the piece wise function which would be the cases of i =1 and i =2 which both equating to 1 which gives use that the sumation of F of i from i=1 to i=2 is 2}

{Now lets solve for i=3 to n, since this part takes the functions previous result and adds it to the second previous result this would give us f sub(n-1) and f sub(n-2)}

{Since we already know that we must add the next two terms starting from the begging of the function the equation equals f sub (n+2) -1 because of the first part of the function}

% ============================================
% ============================================
\collab{n/a} \nextprob{US Coins}
% ============================================
% ============================================

Consider the four smallest denominations of US coins: $D=\{1,5,10,25\}$.  Prove, using
induction, that, for each $n \geq 1$, you can make $n$ cents using at most four
pennies.

\paragraph{Answer}


{Given that D = 1,5,10,25 and through the definition of a factor the numbers 25 and 10 can be represented with 5 because both are divisible by 5 with no remainder.With this we can set up our equation: (n modulus 5) - 4 $ \le $ 0. }

{This equation takes the mod 5 of n which is the max amount of nickles n can be represented with and subtracts 4 which is the maximum amount of pennies that can be used. Since the maximum integer remainder of 5 is 4 and 4 - 4 is indeed equal to 0 we know this equation to be true but lets use induction. Firstly we must represent 0 in terms of n which is n modulus 1 which is always zero because the mod  1 of any integer is always zero. Using this our new equation is (n modulus 5)-4 $ \le $(n modulus 1). To prove our base case we let n = 1, which is (1 mod 5) - 4 $\le $   }


{To prove our base case we let n = 1, which is: }

{(1 mod 5) - 4 $\le $ 1 modulus 1}

{(1)- 4 $\le $ 1}

{-3 $\le $ 1}
{TRUE}


{To prove our all other cases we let n = k prove this is true and prove n= k+1 is true:  }

{(k mod 5) - 4 $\le $ k modulus 1}

{(k mod 5)- 4 $\le $ 1}

{(k mod 5) $\le $ 5}
{TRUE}


{((k+1) mod 5) - 4 $\le $ (k+1) modulus 1}

{((k+1) mod 5)- 4 $\le $ 1}

{((k+1) mod 5) $\le $ 5}
{TRUE}

% ============================================
% ============================================
\collab{n/a} \nextprob{Four Colors Suffice}
% ============================================
% ============================================

Read Chapters $4$ and $5$ of \emph{Four Colors Suffice}.

Use a proof by contradiction to prove that if an edge is removed from a
tree, then the resulting graph has two connected components.

EC:
Use a ``minimal criminal'' argument to prove this.

        \paragraph{Answer}

{A tree is an acyclic graph which means it has no cycles and it is connected. Following this definition a tree has to be a simple graph.}

{We know a graph is a tree if there is absoluty 1 path between all pairs of the vertices}

{Let n = a graph such that there is exactly 1 path between all pairs.}

{Therefore n is connected, meaning no cycles}


{Let a and b be verticies of n}

{Between a and b there are 2 paths which is the contridiction}

{The path between this pair contains a cycle which contradicts one of our proproties of a tree (A tree is an acyclic graph).


% ============================================
% ============================================
\collab{n/a}
\nextprob{Leonhard Euler}
% ============================================
% ============================================

Write a short (1-2 paragraph) biography of Leonhard Euler.
\textbf{In your own words}, describe who they are and why they are important in
the history of computer science.

If you use external resources, please provide
proper citations. If you do not use external sources, please write ``I did not
use any sources to write this biography'' as the last sentence of the
biography.

\paragraph{Answer}

{Leonhard Euler was a swiss mathematician born on April 1707 in Basel Switerzland. Mr. Euler would entrool at the University of Basel where he would recieve a Master of Philosophy. At the start of his professional carrer Euler would arrive at Saint Petersburg in 1727 where he was given a position to the mathematics department from his previous position in the medical department of the academy. Later on in Mr. Euler's professional carrer he would mvoe to Berlin after political turmoil in Russia where he would start a carrer at the Berlin Academy. During his 25 year stay in Berlin he would create the majority of his work which included over 380 articles and 2 of his most renowned works "Introductio in analysin infinitorum" and "Institutiones calculi differentialis" both of which were significant contributions to the math community.  }

{Mr. Euler contributions to mathematics is quite expansive. Firstly Euler is responsible for Euler's identity, the solution to the Basel problem, the gamma function, and the zeta function. These are all significant to mathematics but the most significant contribution to computer science is none of these rather it is one of his most renowned works "Institutiones calculi differentialis" which translates to introduction to differentiable calculus which is used in modern day computer science. A few applications of calculus in computer science include machine learning, data mining, scientific computing, image processing, and graphic/physic engines used in video games. To give an example without the contributions of Mr. Eulers work to calculus it would be significantly more difficult to create 3D maps of virtual objects that often require laplace forumulas. Overall Mr. Euler is one of the most significant contributors to mathematics, much of wich have a significant impact on computer science such as calculus.

}

% %% ... the bibliography
\newpage
\bibliographystyle{acm}
\bibliography{biblio}
\paragraph{Citation: AE Networks Television (September 2020) Leonhard Euler}


\end{document}

